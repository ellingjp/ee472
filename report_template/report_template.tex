% !TEX TS-program = pdflatex
% !TEX encoding = UTF-8 Unicode

% This is a simple template for a LaTeX document using the "article" class.
% See "book", "report", "letter" for other types of document.

\documentclass[12pt]{article} % use larger type; default would be 10pt
\usepackage[utf8]{inputenc}   % set input encoding (not needed with XeLaTeX)

%%% PAGE DIMENSIONS
\usepackage{geometry}
\geometry{a4paper}
\geometry{margin=1in} % 1in page margin

%%% COLOR AND GRAPHICS
\usepackage{color}
\usepackage{graphicx} % support the \includegraphics command and options

\usepackage{pslatex}
\definecolor{mygreen}{rgb}{0,0.6,0}
\definecolor{mygray}{rgb}{0.5,0.5,0.5}
\definecolor{mymauve}{rgb}{0.58,0,0.82}
\usepackage{listings} % For displaying source code
\lstset{ %
  language=C,                      % the language of the code
  backgroundcolor=\color{white},   % choose the background color; you must add \usepackage{color} or \usepackage{xcolor}
  basicstyle=\footnotesize,        % the size of the fonts that are used for the code
  breakatwhitespace=false,         % sets if automatic breaks should only happen at whitespace
  breaklines=true,                 % sets automatic line breaking
  captionpos=t,                    % sets the caption-position to bottom
  commentstyle=\color{mygreen},    % comment style
  deletekeywords={...},            % if you want to delete keywords from the given language
  escapeinside={\%*}{*)},          % if you want to add LaTeX within your code
  extendedchars=true,              % lets you use non-ASCII characters; for 8-bits encodings only, does not work with UTF-8
  frame=single,                    % adds a frame around the code
  keepspaces=true,                 % keeps spaces in text, useful for keeping indentation of code (possibly needs columns=flexible)
  keywordstyle=\color{blue},       % keyword style
  morekeywords={*,...},            % if you want to add more keywords to the set
  numbers=left,                    % where to put the line-numbers; possible values are (none, left, right)
  numbersep=5pt,                   % how far the line-numbers are from the code
  numberstyle=\color{mygray},      % the style that is used for the line-numbers
  rulecolor=\color{black},         % if not set, the frame-color may be changed on line-breaks within not-black text (e.g. comments (green here))
  showspaces=false,                % show spaces everywhere adding particular underscores; it overrides 'showstringspaces'
  showstringspaces=false,          % underline spaces within strings only
  showtabs=false,                  % show tabs within strings adding particular underscores
  stepnumber=1,                    % the step between two line-numbers. If it's 1, each line will be numbered
  stringstyle=\color{mymauve},     % string literal style
  tabsize=2,                       % sets default tabsize to 2 spaces
  title=\lstname                   % show the filename of files included with \lstinputlisting; also try caption instead of title
}

% \usepackage[parfill]{parskip} % Activate to begin paragraphs with an empty line rather than an indent

%%% PACKAGES
\usepackage{booktabs} % for much better looking tables
\usepackage{array}    % for better arrays (eg matrices) in maths
\usepackage{paralist} % very flexible & customisable lists (eg. enumerate/itemize, etc.)
\usepackage{verbatim} % adds environment for commenting out blocks of text & for better verbatim
\usepackage{subfig}   % make it possible to include more than one captioned figure/table in a single float

%%% HEADERS & FOOTERS
%\usepackage{fancyhdr} % This should be set AFTER setting up the page geometry
%\pagestyle{fancy} % options: empty , plain , fancy
%\renewcommand{\headrulewidth}{0pt} % customise the layout...
%\lhead{}\chead{}\rhead{}
%\lfoot{}\cfoot{\thepage}\rfoot{}


%%% SECTION TITLE APPEARANCE
\usepackage{sectsty}
\sectionfont{\normalsize\bfseries\uppercase}
\subsectionfont{\normalsize\bfseries}
\subsubsectionfont{\normalsize\mdseries\itshape}

%%% ToC (table of contents) APPEARANCE
\usepackage[nottoc,notlof,notlot]{tocbibind} % Put the bibliography in the ToC
\usepackage[titles,subfigure]{tocloft} % Alter the style of the Table of Contents
\renewcommand{\cftsecfont}{\rmfamily\mdseries\upshape}
\renewcommand{\cftsecpagefont}{\rmfamily\mdseries\upshape} % No bold!

%%% Title setup
\newcommand{\TitleFont}{\fontsize{16}{20}\selectfont\bfseries}
\newcommand{\AuthorFont}{\fontsize{14}{17}\selectfont}

%%% END Article customizations

%%% The "real" document content comes below...

\title{\TitleFont EE 472 Lab 1 \\ Introducing the Lab Environment \vfill }
\author{\AuthorFont Jonathan Ellington \\ Patrick Ma \\ Jarrett Gaddy}
\date{}

\begin{document}

%% Make title and ToC, start page numbering AFTER ToC
\maketitle
\thispagestyle{empty}
\pagebreak
\tableofcontents
\listoftables
\listoffigures
\thispagestyle{empty}
\pagebreak
\setcounter{page}{1}

\section{Introduction}

The introductory lab contains 3 parts to introduce the students to different parts of working with the Stellaris embedded system. Part 1 of the lab gives the students a piece of sample code that will print a few numbers on an OLED screen on the Stellaris board. The lab asks the students to modify the code in a number of ways to implement the same code in a slightly different way. Examples of what the lab asks for are modifying the program to use function calls. Part 2 of the lab introduces the students to debugging their code that is currently running on the Stellaris board using the IAR program’s debugging capabilities. 2 programs with minor issues are provided to be run on the board while the debugger is being used to find and fix the problems. Finally for part 3 of the lab the students must use what they have learned from part 1 and 2 to create their own applications that will display messages on the OLED display on the Stellaris board. The 3 applications each involve blinking the characters A, B, C, and D in different patterns.

The introductory lab contains 3 parts to introduce the students to different parts of working with the Stellaris embedded system. Part 1 of the lab gives the students a piece of sample code that will print a few numbers on an OLED screen on the Stellaris board. The lab asks the students to modify the code in a number of ways to implement the same code in a slightly different way. Examples of what the lab asks for are modifying the program to use function calls. Part 2 of the lab introduces the students to debugging their code that is currently running on the Stellaris board using the IAR program’s debugging capabilities. 2 programs with minor issues are provided to be run on the board while the debugger is being used to find and fix the problems. Finally for part 3 of the lab the students must use what they have learned from part 1 and 2 to create their own applications that will display messages on the OLED display on the Stellaris board. The 3 applications each involve blinking the characters A, B, C, and D in different patterns.

\subsection{A subsection}

Some more text and a table. \\

\begin{table}[h]
    \centering
    \label{table:ex}
    \begin{tabular}{llr}
        \toprule
        Animal    & Description & Price (\$) \\
        \midrule
        Gnat      & per gram    & 13.65      \\
                  & each        & 0.01       \\
        Gnu       & stuffed     & 92.50      \\
        Emu       & stuffed     & 33.33      \\
        Armadillo & frozen      & 8.99       \\
        \bottomrule
    \end{tabular}
    \caption{Example table.}
\end{table}


\subsubsection{A subsubsection}

Even more text. This shows that \LaTeX's figure positioning isn't perfect, and we'll have to work around it in some cases.

\begin{figure}[h]
    \centering
    \label{figure:chick}
    \includegraphics{figure1.png}
    \caption{A chick}
\end{figure}

\section{Yet another section}

\pagebreak
\appendix

\section{Source Code}
\subsection{The first part}
\lstinputlisting{source.c}

\subsection{The second part}
\lstinputlisting{source.c}

\end{document}
