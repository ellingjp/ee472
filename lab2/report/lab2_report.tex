% !TEX TS-program = pdflatex
% !TEX encoding = UTF-8 Unicode

% This is a simple template for a LaTeX document using the "article" class.
% See "book", "report", "letter" for other types of document.

\documentclass[12pt]{article} % use larger type; default would be 10pt
\usepackage[utf8]{inputenc}   % set input encoding (not needed with XeLaTeX)

%%% PAGE DIMENSIONS
\usepackage{geometry}
\geometry{a4paper}
\geometry{margin=1in} % 1in page margin

%%% COLOR AND GRAPHICS
\usepackage{color}
\usepackage{graphicx} % support the \includegraphics command and options

\usepackage{pslatex}
\definecolor{mygreen}{rgb}{0,0.6,0}
\definecolor{mygray}{rgb}{0.5,0.5,0.5}
\definecolor{mymauve}{rgb}{0.58,0,0.82}
\usepackage{listings} % For displaying source code
\lstset{ %
  language=C,                      % the language of the code
  backgroundcolor=\color{white},   % choose the background color; you must add \usepackage{color} or \usepackage{xcolor}
  basicstyle=\sffamily\fontsize{11}{13.2}\selectfont,        % the size of the fonts that are used for the code
  breakatwhitespace=false,         % sets if automatic breaks should only happen at whitespace
  breaklines=true,                 % sets automatic line breaking
  captionpos=t,                    % sets the caption-position to bottom
  commentstyle=\color{mygreen},    % comment style
  deletekeywords={...},            % if you want to delete keywords from the given language
  escapeinside={\%*}{*)},          % if you want to add LaTeX within your code
  extendedchars=true,              % lets you use non-ASCII characters; for 8-bits encodings only, does not work with UTF-8
  frame=single,                    % adds a frame around the code
  keepspaces=true,                 % keeps spaces in text, useful for keeping indentation of code (possibly needs columns=flexible)
  keywordstyle=\color{blue},       % keyword style
  morekeywords={*,...},            % if you want to add more keywords to the set
  numbers=left,                    % where to put the line-numbers; possible values are (none, left, right)
  numbersep=5pt,                   % how far the line-numbers are from the code
  numberstyle=\color{mygray},      % the style that is used for the line-numbers
  rulecolor=\color{black},         % if not set, the frame-color may be changed on line-breaks within not-black text (e.g. comments (green here))
  showspaces=false,                % show spaces everywhere adding particular underscores; it overrides 'showstringspaces'
  showstringspaces=false,          % underline spaces within strings only
  showtabs=false,                  % show tabs within strings adding particular underscores
  stepnumber=1,                    % the step between two line-numbers. If it's 1, each line will be numbered
  stringstyle=\color{mymauve},     % string literal style
  tabsize=2,                       % sets default tabsize to 2 spaces
  title=\lstname                   % show the filename of files included with \lstinputlisting; also try caption instead of title
}

% \usepackage[parfill]{parskip} % Activate to begin paragraphs with an empty line rather than an indent

%%% PACKAGES
\usepackage{booktabs} % for much better looking tables
\usepackage{array}    % for better arrays (eg matrices) in maths
\usepackage{paralist} % very flexible & customisable lists (eg. enumerate/itemize, etc.)
\usepackage{verbatim} % adds environment for commenting out blocks of text & for better verbatim
\usepackage{subfig}   % make it possible to include more than one captioned figure/table in a single float

%%% HEADERS & FOOTERS
%\usepackage{fancyhdr} % This should be set AFTER setting up the page geometry
%\pagestyle{fancy} % options: empty , plain , fancy
%\renewcommand{\headrulewidth}{0pt} % customise the layout...
%\lhead{}\chead{}\rhead{}
%\lfoot{}\cfoot{\thepage}\rfoot{}


%%% SECTION TITLE APPEARANCE
\usepackage{sectsty}
\sectionfont{\normalsize\bfseries\uppercase}
\subsectionfont{\normalsize\bfseries}
\subsubsectionfont{\normalsize\mdseries\itshape}

%%% ToC (table of contents) APPEARANCE
\usepackage[nottoc,notlof,notlot]{tocbibind} % Put the bibliography in the ToC
\usepackage[titles,subfigure]{tocloft} % Alter the style of the Table of Contents
\renewcommand{\cftsecfont}{\rmfamily\mdseries\upshape}
\renewcommand{\cftsecpagefont}{\rmfamily\mdseries\upshape} % No bold!

%%% Title setup
\newcommand{\TitleFont}{\fontsize{16}{20}\selectfont\bfseries}
\newcommand{\AuthorFont}{\fontsize{14}{17}\selectfont}

%%% END Article customizations

%%% The "real" document content comes below...

\title{\TitleFont EE 472 Lab 1 \\ Introducing the Lab Environment \vfill }
\author{\AuthorFont Jonathan Ellington \\ Patrick Ma \\ Jarrett Gaddy}
\date{}

\begin{document}

%% Make title and ToC, start page numbering AFTER ToC
\maketitle
\thispagestyle{empty}
\pagebreak
\tableofcontents
\listoftables
\listoffigures
\thispagestyle{empty}
\pagebreak
\setcounter{page}{1}

\section{Abstract}
The abstract should provide a brief overview of the report.  It should provide a summary of the main specific points for the introduction, the main tests and experiments, the results, and the conclusions. It is called an abstract because you can literally "abstract" sentences from the other sections. 

Once again, this is not a narrative of your experiences as you executed the design.  The abstract should mirror (albeit in a very condensed way) the content of your report.

\section{Introduction}
\textbf{Brief introduction and overview of the purpose of the lab and of the methods and tools used.}

\section{Discussion of the Lab}

This section should include the following:

\subsection{Design Specification\label{sec:designSpec}}

\subsubsection{Specification Overview}
The entire system must satisfy several lofty objectives. The final product must be portable, lightweight, and internet enabled. The system must also make measurements of vital bodily functions, perform simple computations, provide datalogging functionality, and indicate when measured vitals exceed given ranges, or the user fails to comply with a prescribed logging regimen. \\
At the present time, only two subsystems must be produced: the display and alarm portions. Additionally, the system must demonstrate the ability to store basic measurements. \\

\begin{itemize}[$$]
  \item The initial functional requirements for the system are:
    \begin{itemize}[$\bullet$]
      \item Provide continuous sensor monitoring capability
      \item Produce a visual display of the sensor values
      \item Accept variety of input data types
      \item Provide visual indication of warning states
      \item Provide an audible indicator of alarm states
    \end{itemize}
\end{itemize}

\subsubsection{Detailed Specifications}
For this project, these requirements have been further specified as follows:

\begin{itemize}[$$]
  \item The system must have the following inputs:
    \begin{itemize}[$\bullet$]
      \item Alarm acknowledgment capability using a pushbutton
      \item Sensor measurement input capability consisting of:
	\begin{itemize}
	\item Body temperature measurement
	\item Pulse rate measurement
	\item Systolic blood pressure measurement
	\item Diastolic blood pressure measurement
	\end{itemize}
    \end{itemize}
\end{itemize}


\begin{itemize}[$$]
 \item The system must have the following outputs:
    \begin{itemize}[$\bullet$]
      \item Visual display of the following data in human-readable formats:
	\begin{itemize}
	\item Body temperature
	\item Pulse rate
	\item Systolic blood pressure
	\item Diastolic blood pressure
	\item Battery status
	\end{itemize}
      \item Visually indicate warning state with a flashing LED
      \item Visually indicate a low battery state with an LED
      \item Audibly indicate an alarm state using a speaker
    \end{itemize}
\end{itemize}

The initization values, normal measurement ranges, displayed units, and 
warning and alarm behaviors for each vital measurement are given in 
Table~\ref{tab:sensorDefs}. The sensors must be sampled every five seconds.\\

\begin{tabular}{|l|*{5}{c}|}
	\hline
	Measurement & Units & Initial Value & Min. Value & Max. Value & Warning Flash Period \\ \hline
	Body Temperature & C & 75 & 36.1C & 37.8C & 1 sec \\ \hline
	Systolic BP  & mm Hg & 80 & - & 120 mmHg & 0.5 sec \\ \hline
	Diastolic BP & mm Hg & 80 & - & 80mmHg & 0.5 sec \\ \hline
	Pulse Rate & BPM & 50 & 60 BPM & 100 BPM & 2 sec \\ \hline
	Remaining Battery & \% & 200 & 40~\% & - & Constant \\ \hline
\end{tabular}

A measurement enters a warning state when its value falls outside the stated 
normal range by 5\%. An alarm state occurs when any measurement falls outside
 its stated normal range by 10\%.

Additionally, the system must be implemented using the Stellaris 
EKI-LM3S8962 ARM Cortex-M3 microcomputer board, The software for the system 
must be written in C using the IAR Systems Embedded Workbench/Assembler IDE.



\subsubsection{Identified Use Cases}
Taking the functional requirements listed above, several use cases were
 developed. A Use case diagram of these scenarios is given in Figure~\ref{fig;useCases}. Each use case is expanded and explained below.

\begin{figure}[h]
	\centering
	\includegraphics[width=\textwidth]{../design/use_cases_graphical.png}
	\caption{Use case diagram}
	\label{fig:useCases}
\end{figure}

\textbf{Use Case \#1: View Vital Measurements } \\
In the first use case, the user views the basic measurements picked up by the
sensors connected to the device. \\
During normal operation, once the device is turned on by the user, the system
records the value output by each sensor. This raw value is linearized and 
converted into a human-readable form. Finally, this value is displayed onscreen. \\

Three exceptional conditions were identified for this use case: 
\begin{description}
\item [One or more of the expected sensors is not connected:] If this occurs, the measurements taken by the device may be erratic. At the present moment, no action will be taken in such events. Later revisions may address the issue
\item[A measured value is outside 5\% of the specified normal range:] In this case, a warning signal will flash as an indication of the warning condition
\item[A measured value falls outside 10\% of a specified "normal" range:] In this case, an audible alarm will sound to indicate the alarm condition
\end{description}

\textbf{Use Case \#2: Acknowledge Alarm} \\
In the second case, the system is in an alarm state. The user acknowledges
the alarm condition by pressing a button. \\
Upon pressing the button, the system silences the audible alarm. Any visual warnings continue to flash during the silenced period. If a specified amount 
time passes and the sensor reading(s) continue to maintain an alarmed state,
the audible alarm will recommence.\\ 

No exceptional conditions were identified for this use case.\\


\subsection{Software Implementation  Jon}


\subsubsection{Top level design Jon}

%- User use cases  (need to fix)
%  + Take measurements
%  + Acknowledge alarm
%- High level blocks
%  + Functional decomposition
%    o User
%    o Stellaris board
%    o External Sensors
%  + System Architecture (need to flip inheritance arrows and composition arrow)
%    o Discuss shared data
%    o Discuss TCB->Schedular composition
%    o Discuss TCB->task inheritance
%    o Interaction with hardware
%- Implementation in C
%  + Scheduler
%    o Has queue of TCBs
%    o Runs each with minor cycle delay
%      - Timebase
%        o Specifies major/minor cycle
%  + Tasks
%    o Global data is declared in a header file, globals.h and shared with everyone
%    o Get their own file and header file
%    o Own data is hidden from rest of program, single pointer exposed
%    o Every task gets an initialization to initialize data

The design process began by identifying the use cases and actors involved with the system.  In the specified system, the user interacts with the system in one of two ways:
\begin{enumerate}
  \item The user can view their vitals
  \item The user will acknowledge an alarm condition to silence the alarm
\end{enumerate}
In order for a user to view their vitals, the system will have to interact with some external sensors.  Specifically, the system will interact with blood pressure, temperature, and pulse rate sensors.  A graphical depiction of this is shown in Figure~\ref{fig:use}.

\begin{figure}
    \centering
    \includegraphics[width=\textwidth]{../design/use_cases_graphical}
    \caption{Use-Case Diagram}
    \label{fig:use}
\end{figure}

After understanding how the user would interact with the device, the system was
functionally decomposed into high-level blocks as shown in
Figure~\ref{fig:func}.  The main system control is located in the CPU, which
controls all data flow into and out of the peripheral devices.  The OLED
displays the user's current vitals including blood pressure (systolic and
diastolic), temperature, and pulse rate.  In the future external sensors will
be added, but for now the values are simulated using the CPU.  The CPU also
controls three LEDs colored green, yellow, and red.  These LEDs are used to
inform the user on the current state of their vitals as well as the state of
the device.  Under normal circumstances, the green LED will be lit.  If the
users' vitals fall outside of a specified range, the red LED will flash at a
specified rate, dependant on which vital is out of range.  If the battery is
low, the yellow LED will be illuminated.

\begin{figure}[h]
    \centering
    \includegraphics[width=\textwidth]{../design/Functional_decomposition}
    \caption{Functional Decomposition}
    \label{fig:func}
\end{figure}

Next, the system architecture was developed (Figure~\ref{fig:arch}).  At a high
level the system works on two main concepts, the scheduler and tasks.  Tasks
embody some sort of work being done, and the scheduler is in charge of
determining the speed and order in which the tasks execute.  The system has
several tasks, each with their own specific job.  For modularity reasons, each
task should have the same public interface and the scheduler should be able to
run each task regardless of that specific tasks job or implementation.  Thus
the task concept is abstracted into a Task Control Block (TCB), and the
scheduler maintains a queue of TCBs to run.  The TCB abstraction is shown in
Figure~\ref{fig:arch} using inheritance, and the fact that the scheduler has a
queue of TCBs is shown with composition.  The core functionality of the system
was divided into the following five main tasks:
\begin{itemize}
  \item \textbf{Measure Task} - In charge of interacting with the blood pressure, temperature, and pulse sensors (simulated)
  \item \textbf{Compute Task} - Converts sensor data into human readable format
  \item \textbf{Display Task} - Displays the measurements on the Stellaris OLED
  \item \textbf{Warning/Alarm Task} - Interacts with the red, yellow, and green LEDs, as well as the speaker to annunciate warning and alarm information
  \item \textbf{Status Task} - Receives battery information from the device
\end{itemize}
Each of these tasks interact using the shared data shown in Figure~\ref{fig:arch}. 

\begin{figure}
    \centering
    \includegraphics[width=\textwidth]{../design/System_Architecture}
    \caption{System Architecture Diagram}
    \label{fig:arch}
\end{figure}

After developing the system architecture, the design needed to be translated into the C programming language.  The design manifested in a multifile program consisting of the following source files:
\begin{itemize}
  \item \textbf{globals.c/globals.h} - Used to define the Shared Data used among the tasks
  \item \textbf{schedule.c/schedule.h} - Defines the scheduler interface and it's implementation
  \item \textbf{timebase.h}
\end{itemize}

\subsubsection{low level design  Jarret}

implementation details here. Tasks, scheduler, etc. Control diagram goes here, activity diagram, etc.

\section{Presentation, Discussion, and Analysis of the Results}

Based upon the execution of your design, present your results. Explain them and what was expected, and draw any conclusions (for example, did this prove your design worked).

In addition to a detailed discussion and analysis of your project and your results, you must include all the answers to all questions raised in the lab.
\subsection{results  patrick}
The project was completed and demonstrated on January 29, 2014. 



\subsection{discussion of results  Jon}

\subsection{Analysis of any Errors}
The project was completed without any residual errors or unsolved problems. See the following section for analysis of issues encountered during the project.

\subsection{Analysis of problems and issues encountered and what efforts were made to identify the root cause of any problems  Jarrett}

State any problems you encountered while working on the project. If your project did not work or worked only partially, provide an analysis of why and what efforts were made to identify the root cause of any problems. \\

Some points to bring up: did not enable the GPIO bank (caused OLED display to not work), could not get switch to work (solved by understanding that switch required pull up). P or J can talk about design solutions that did not work. On the whole, we had problems with going too deep, too quickly.

\section{Test Plan}

To ensure that this project meets the specifications listed in 
section~\ref{sec:designSpec}, the following parts of the system must be 
tested: 

\begin{itemize}
	\item Vitals are measured and updated every five seconds
	\item System properly displays corrected measurements and units properly
	\item System enters, indicates, and exits the proper warning state for blood pressure, temperature, pulse, and battery
	\item System enters and exits the alarm state correctly
	\item Alarm is silenced upon button push
	\item Alarm recommences sound after silencing if system remains in alarm state longer than slilence period
\end{itemize}

Additional tests to determine the runtime of each specific task are also required.

\subsection{Test Specification}

Annotated description of what is to be tested and the test limits.  This specification quantifies inputs, outputs, and constraints on the system.  That is, it provides specific values for each. 

Note, this does not specify test implementation...this is what to do, not how to do it.

\subsection{Test Cases  Jarrett}

Annotated description of how your system is to be tested against the test limits
Note, this does specify test implementation...this is not what to do, this is how to do it based upon the test specification.

\section{Summary and Conclusion}

You should know these sections very well, no need to explain.  Note, however, that they are two different sections.  The summary is just that, a summary of your project.  It should loosely mirror the abstract with a bit more detail.  The conclusion concludes the report, potentially adds information that is often outside the main thrust of the report, and may offer suggestions or recommendations about the project.

\subsection{Final Summary}


\subsection{Project Conclusions}


\pagebreak
\appendix

\section{Breakdown of Lab Man-hours (Estimated)}

\begin{tabular}{|l|*{4}{r|}}
	\hline
	Person & Design Hrs & Code Hrs & Test/Debug Hrs & Documentation Hrs \\ \hline
	Patrick & 15 & 4 & 2 & 9  \\ \hline
	Jarret & x & x & x & x  \\ \hline
	Jonathan & x & x & x & x  \\ \hline
\end{tabular}

~\\

By initializing/signing above, I attest that I did in fact work the estimated number of hours stated. I also attest, under penalty of shame, that the work produced during the lab and contained herein is actually my own (as far as I know to be true). If special considerations or dispensations are due others or myself, I have indicated them below.

\pagebreak

\section{Source Code}

Source code for this project is provided below.

\subsection{The first part}
% \lstinputlisting{source.c}

\subsection{The second part}
% \lstinputlisting{source.c}

\end{document}
